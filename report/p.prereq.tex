\section{Pre-Requisite Software}
\label{prereq}

Many efforts were made for the toolbox and techniques used in this learning
module to not rely on any non-default software, alas, some snuck in.
This learning module requires some python modules, and ArcGIS to be
upgraded to Service Pack 1.

The required python module is called the \emph{MultipartPostHandler} and is
used to transfer the IOAPI to and from the server (purpose described
in the appendix \ref{ioapi_tool_detail}.)  This module however is only
available as a python \emph{egg}, which requires their own library to
be installed.

Therefore, it is recommended the following steps be followed.
Remember throughout these instructions that ArcGIS only supports
python 2.6, so if ever confronted with a choice of package to download
based on python version, choose 2.6.
\begin{enumerate}
	%\item Download \emph{easy\_install}, this is available at \url{http://peak.telecommunity.com/dist/ez\_setup.py}
	\item Download \emph{setup tools} \\
	This is described at
\url{http://packages.python.org/an\_example\_pypi\_project/setuptools.html}
and available for download at
\url{http://pypi.python.org/pypi/setuptools} \\
	You will require the file named
\emph{setuptools-0.6c11.win32-py2.6.exe}
	\item Once downloaded, install it.
	\item Ensure that your python is in your path.  If you're using
the python installed with ArcGIS (likely), this likely installed the
setup tools (such as \emph{easy\_install}) into
\texttt{C:/Python26/ArcGIS10.0/Scripts}.  Therefore, this is likely
the path to add to your path variable in \textbf{Control Panel}
$\rightarrow$ \textbf{System} $\rightarrow$ \textbf{Advance}
$\rightarrow$ \textbf{Environment Variables}
	\item Download the \emph{MultipartPostHandler} from
\url{http://pypi.python.org/pypi/MultipartPostHandler/}
	\item Open the command prompt, and install the \emph{egg} file for
this module with the \emph{easy\_install} command.
	\item Download ArcGIS v10 Service Pack 1 from
\url{http://resources.arcgis.com/content/patches-and-service-packs?fa=viewPatch\&PID=15\&MetaID=1685#install-Windows}
\\
	Note, this takes considerably longer than you might expect to
install.
\end{enumerate}

