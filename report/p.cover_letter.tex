
\pagestyle{empty}

\noindent Dan Patterson\\
Room B340A LA\\
Carleton University\\
1125 Colonel By Drive\\
Ottawa, Ontario\\
K1S 5B6 Canada\\


\vspace{5mm}
\noindent Dear Dan Patterson,
\vspace{5mm}

The enclosed document is a learning module entitled ``Using ArcGIS to
Aid Setup and Evaluation of Air Quality Models'' submitted to
satisfy the project component of the fall 2011 session of the GEOG
5804 - Introduction to GIS course offered by the department of Arts
and Social Science at Carleton University.

The learning module enclosed is designed to help open up the ArcGIS
community tools to individuals involved in air quality modelling.  The
crowning achievement of this document, in my opinion, is the detailed
description of the development of a tool that allows modellers to
import their \ioapi~formatted output files into ArcGIS via a one step
process.  To my knowledge, no one has successfully done this before.
And therefore, more than a school project, this document could qualify
as a community contribution.  I believe this statement is further reinforced
by how another project was already effectively enabled by the work done on
this project even before completion.

The style of the learning module is that of a ArcGIS tutorial.  As such, a
first person - sometimes casual - tone was adopted and a basic
competence with ArcGIS was assumed.

Though this learning module is far from what the original project
proposal would suggest, I believe it does met the objective of the
project criteria, specifically, this learning module demonstrates my
use and comprehension of:
\begin{itemize}
	\item the multidimensional toolbox
	\item tools and techniques for point analysis
	\item working with and defining projections and spatial reference frames
	\item developing custom tools, complete with full tool descriptions
	\item using third-party community resources
\end{itemize}
Moreover, this project has been an exciting opportunity for me to
learn Python and associated technologies (\emph{mod\_python}), become
substantially more familiar with the \netcdf~file format and
increase my familiarity with ArcGIS.

The reason for the substantial change in purpose and scope of the
project relative to the proposal was that when the proposal was
written I was not familiar enough with ArcGIS to properly estimate the
effort involved in what was being proposed.  More than anything else,
I was figuratively blindsided by my knowledge gap of how spatial data
is defined in \netcdf~files, how far off \ioapi~files were from that,
and the methods required to transform \ioapi~files.

Along with that note, it is only fair to mention that this project was
challenged by time lines, as such, comprehensive quality assurance
techniques have not been completed on the tools developed for this
learning module, and the code developed could use more complete
comments.  Therefore, it is possible (though unlikely!) that some use
cases that the code below was developed for may fail for unexpected
reasons.

For any questions concerning this learning module, please contact me at 
\href{mailto:mrussel2@connect.carleton.ca}{mrussel2@connect.carleton.ca}.

\vspace{10mm}
\noindent Sincerely,

\vspace{15mm}
\noindent Matthew Russell (100298305) \\
Room 3346 ME\\
Carleton University\\
1125 Colonel By Drive\\
Ottawa, Ontario\\
K1S 5B6 Canada\\
\vspace{5mm}

\noindent Encl: Learning Module: ``Using ArcGIS to Aid Setup and Evaluation of Air Quality Models''
%\end{letter}


