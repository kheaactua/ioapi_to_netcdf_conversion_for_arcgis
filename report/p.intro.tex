This learning module is aimed towards students and researchers who
work with \acf{aqm}.  Specifically, this learning module introduces a
tool that has been developed to import \ac{aqm} output into ArcGIS.

Air quality modelling involves assimilating meteorological, emission,
land use and solar data and running \acfp{ctm} to model the
concentration of gas and particle species that can be harmful to human
health.  Though the techniques discussed in this learning module are
applicable to almost any model, it is particularly aimed at those who
use the \acs{usepa}'s model \ac{cmaq}.  This is because the main
challenge that this learning module addresses is specific to the file
format output by \ac{cmaq}

The \ac{aqm} community is one that is growing every year as air
quality issues become more relevant to every day life.  One factor
that contributes to the growth of this - or any - community is the
availability of tools to aid ones research.  In this authors
experience, a few large software tools are developed by large
governmental organizations such as the \acs{usepa} which serve as the
backbone to the modelling process, then small research group develop
smaller innovative tools that suits their specific needs.

For example, \acs{bams} produces \ioapi~-a \netcdf~wrapper layer -
which is used for all inputs and outputs to \ac{cmaq}, and then an
individual research group will develop automation scripts to
manipulate the inputs and outputs in any given way, and send the data
to some visualization package.

\ac{cmaq} - for better or for worse - has chosen the aforementioned file format,
\ioapi, that cannot properly be imported into GIS programs.  The
result is that modellers using the \ioapi~format are therefore cut off
form the ArcGIS community tools - a vast library of tools to aid in
anything from data manipulation to data analysis.  This learning
module introduces a tool that can convert \ioapi~files to a form that
can be read by ArcGIS, thus bridging the gap between the two
technologies and perhaps even the two communities.

This learning module reads as a tutorial where the overall goal is to
use ArcGIS to aid in domain set up to model evaluation.  To follow
this learning module, some non-standard prerequisite software is
however required (python libraries and ArcGIS upgrades), this is
described in section \ref{prereq}

Section \ref{define_grids} walks the reader through creating the
domain they wish to model.  Notes on what makes good domain choices as
well as instructions on defining your projection are presented.

Section \ref{import_ioapi_data} begins by assuming that the reader has
used the domains set up in section \ref{define_grids} and now have
\ioapi~output files to import into ArcGIS.  This section introduces
the tool to import \ioapi~data.

Section \ref{eval} briefly introduces basic techniques of model
evaluation using ArcGIS.  This section is by no means comprehensive,
and is only intended to provide the reader with an idea of the
possibilities.

